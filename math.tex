%!TEX root = introduction.tex
\DeclareMathOperator{\sgn}{sgn} % Signum
\DeclareMathOperator{\im}{im} % Bild
\DeclareMathOperator{\supp}{supp} % Träger/Support
\DeclareMathOperator{\id}{id} % Identität
\DeclareMathOperator{\pr}{pr} % Projektion
\DeclareMathOperator{\Hom}{Hom} % Homomorphismen
\DeclareMathOperator{\sym}{sym} % symmetrisch
\DeclareMathOperator{\Skew}{skew} % schiefsymmetrisch
\DeclareMathOperator{\coker}{coker}
% \DeclareMathOperator{\deg}{deg}
% \DeclareMathOperator{\zentrum}{Z}
\newcommand{\zentrum}{\mathrm{Z}}

% -- Matrixgruppen
\DeclareMathOperator{\Mat}{Mat} % Matrizen
\DeclareMathOperator{\End}{End} % Endomorphismen
\DeclareMathOperator{\GL}{GL} % allgemeine lineare Gruppe
\DeclareMathOperator{\SL}{SL} % spezielle lineare Gruppe
\DeclareMathOperator{\On}{O} % orthogonale Gruppe
\DeclareMathOperator{\SO}{SO} % spezielle orthogonale Gruppe
\DeclareMathOperator{\Un}{U} % unitäre Gruppe
\DeclareMathOperator{\SU}{SU} % spezielle unitäre Gruppe

% -- besserer underbrace befehl
\newcommand{\Underbrace}[2]{{\underbrace{#1}_{#2}}}


% -- Zum Finetuning von Befehlen
\makeatletter
\newcommand{\raisemath}[1]{\mathpalette{\raisem@th{#1}}}
\newcommand{\raisem@th}[3]{\raisebox{#1}{$#2#3$}}
\makeatother


%--Mengen
\newcommand\SetSymbol[1][]{\nonscript\:#1\vert\allowbreak\nonscript\:\mathopen{}}
\providecommand\given{} % to make it exist
\DeclarePairedDelimiterX\set[1]\{\}{\renewcommand\given{\SetSymbol[\delimsize]}#1}

% -- Betrag und Skalarprodukt
\DeclarePairedDelimiter{\abs}{\lvert}{\rvert}
\DeclarePairedDelimiterX\skal[2]{\langle}{\rangle}{#1\,,\,#2}

%--Umklammern
\DeclarePairedDelimiter\enbrace{(}{)}
\DeclarePairedDelimiter\benbrace{[}{]}
\DeclarePairedDelimiter\homo{\llbracket}{\rrbracket}
\newcommand{\ssbrace}[1]{{\scriptscriptstyle\enbrace{#1}}}

%--Norm
\DeclarePairedDelimiter\norm{\Vert}{\Vert}

% -- Box mit vorgegebener minimaler Länge
\DeclareRobustCommand{\minwidthbox}[2]{%
  \ifmmode
    \expandafter\mathmakebox
  \else
    \expandafter\makebox
  \fi
  [\ifdim#2<\width\width\else#2\fi]{#1}%
}

% -- farbiges Untersteichen im Mathe-Modus
\def\mathul#1#2{\color{#1}\underline{{\color{black}#2}}\color{black}} 


% - verbessertes Integral
\newcommand{\Int}[1]{\int_{\mathrlap{#1}}\,\,}

% -- Alles spezielles aus der Differentialtopologie
\newcommand{\Tang}{\ensuremath{\mathrm{T}\mkern-0.85mu}}
\newcommand{\mathd}{\ensuremath{\mathrm{d}\mkern-0.7mu}}
\newcommand{\diff}[2]{\ensuremath{\frac{{\partial #1}}{{\partial #2}}}}
\newcommand{\diffs}[2]{\ensuremath{\partial #1/\partial #2}}
\newcommand{\diffd}[2]{\ensuremath{\frac{\mathd #1}{\mathd #2} }}
\DeclareMathOperator{\Lie}{L}
\DeclareMathOperator{\Ad}{Ad}  
\DeclareMathOperator{\ad}{ad}
\DeclareMathOperator{\Spin}{Spin}
% \DeclareMathOperator{\spin}{spin}
\newcommand{\spin}{\mathop{\mathfrak{spin}}}
\newcommand{\Ce}{\mathcal{C}}
\DeclareMathOperator{\vol}{vol}


% -- Kategorien und Ähnliches
\newcommand{\VEKT}{\textsc{Vekt}}

%--Abbildungsdefinition
\newcommand{\mapdef}[5]{%
	\[
		\begin{array}{rcl}
			\textstyle #1 &\xrightarrow{\minwidthbox{#5}{2em}} & \textstyle #2 \\[0.5ex]
			\textstyle #3 &\xmapsto{\minwidthbox{\mbox{ }}{2em}} & \textstyle #4
		\end{array}
	\]
}

%--modifiziertes Stackrel
\newcommand{\StackText}[2]{\stackrel{\mbox{\scriptsize #1}}{#2}}
\newcommand{\StackTextClap}[2]{\stackrel{\mathclap{\mbox{\scriptsize #1}}}{#2}}
